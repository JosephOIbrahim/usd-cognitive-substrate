\documentclass[11pt,a4paper]{article}

% Packages
\usepackage[utf8]{inputenc}
\usepackage[T1]{fontenc}
\usepackage{amsmath,amssymb,amsthm}
\usepackage{graphicx}
\usepackage{hyperref}
\usepackage{booktabs}
\usepackage{listings}
\usepackage{xcolor}
\usepackage[margin=1in]{geometry}
\usepackage{natbib}
\usepackage{float}

% Hyperref setup
\hypersetup{
    colorlinks=true,
    linkcolor=blue,
    filecolor=magenta,
    urlcolor=cyan,
    citecolor=blue,
    pdftitle={The Persistent State Hypothesis},
    pdfauthor={Joseph O. Ibrahim}
}

% Listings setup
\lstset{
    basicstyle=\ttfamily\small,
    breaklines=true,
    frame=single,
    backgroundcolor=\color{gray!10}
}

% Title
\title{The Persistent State Hypothesis:\\Challenging the Energy-Intelligence Equivalence Through Composable Knowledge Architectures}

\author{Joseph O. Ibrahim\\
\textit{Independent Research}\\
\url{https://github.com/JosephOIbrahim}}

\date{January 2026}

\begin{document}

\maketitle

\begin{abstract}
The prevailing assumption in artificial intelligence development holds that intelligence necessarily scales with energy consumption---a position recently articulated by DeepMind CEO Demis Hassabis as ``energy will be equivalent to intelligence'' for systems approaching AGI. This paper challenges that assumption, arguing that the energy problem in current AI architectures is \textit{architectural} rather than \textit{fundamental}.

We observe that large language models are stateless prediction engines that recompute from scratch on every inference, and propose that this design choice---not intelligence itself---drives the energy consumption.

We introduce the \textbf{Persistent State Hypothesis}, which posits that emergent capabilities can be preserved in a persistent, composable knowledge substrate that does not require constant recomputation. Drawing on Universal Scene Description (USD) semantics---originally developed for managing computational complexity in visual effects pipelines---we present a theoretical framework for cognitive architectures that treat knowledge as an external environment to navigate rather than content to load entirely into memory.

We report preliminary results from the USD Cognitive Substrate implementation, which demonstrates that USD's composition mechanisms (LIVRPS conflict resolution, payload lazy-loading, layered opinions) successfully manage cognitive behavioral state.

\textbf{Keywords:} Universal Scene Description, cognitive architecture, deterministic AI, state management, neuroplasticity, batch invariance, LIVRPS composition
\end{abstract}

\section{Introduction}

\subsection{The Problem}

At the World Economic Forum in Davos 2026, DeepMind CEO Demis Hassabis characterized the current moment in AI development as ``the most intense competition there has ever been in technology.'' His strategic response centers on a critical assumption: that advancing toward artificial general intelligence (AGI) will require proportionally increasing energy resources. In his formulation, ``energy will be equivalent to intelligence''---an inescapable physical law \cite{hassabis2026davos}.

\textbf{We propose an alternative framing.} The energy problem may be \textit{architectural}, not fundamental. Current large language models are stateless prediction engines---every inference recomputes from scratch, every token of context consumes attention compute, and no derived relationship persists between queries.

\subsection{Contributions}

\begin{enumerate}
    \item \textbf{The Persistent State Hypothesis}: A formal challenge to the energy-intelligence equivalence
    \item \textbf{USD Semantic Mapping}: A theoretical framework mapping USD concepts to cognitive operations
    \item \textbf{Preliminary Implementation}: Results from the USD Cognitive Substrate
    \item \textbf{Research Agenda}: Falsification criteria, probability estimates, and a research roadmap
\end{enumerate}

\section{Background}

\subsection{The Energy-Intelligence Assumption}

The assumption that intelligence scales with energy has both theoretical and empirical foundations. Theoretically: more sophisticated reasoning requires more operations, more operations require more compute, more compute requires more energy.

However, we distinguish between \textit{training} energy and \textit{inference} energy. The scaling laws primarily describe training dynamics \cite{kaplan2020scaling,hoffmann2022chinchilla}. Our hypothesis addresses inference---the cost of answering a query using already-acquired knowledge.

\subsection{Batch Invariance and Deterministic Inference}

Recent work on defeating nondeterminism in LLM inference provides critical infrastructure \cite{thinkingmachines2025}. The key insight: LLM inference nondeterminism stems not from ``concurrency + floating point'' as commonly assumed, but from \textit{batch invariance failures}---the reduction order for each element depends on batch size, which varies with server load.

This finding validates a core premise: \textbf{the apparent randomness in LLM outputs is architectural, not fundamental}. Given identical inputs, the forward pass is deterministic; nondeterminism emerges from system-level choices about batching.

\subsection{Universal Scene Description}

USD is Pixar's framework for managing complex 3D scenes \cite{pixar2016usd}. It solves a problem analogous to ours: how to manage scenes with billions of polygons without recomputing everything constantly.

\textbf{Key mechanisms:}
\begin{itemize}
    \item \textbf{Composition Arcs}: References, payloads, inherits, variants, specializes
    \item \textbf{LIVRPS Resolution}: Deterministic conflict resolution order
    \item \textbf{Lazy Loading}: Payloads defer loading until needed
    \item \textbf{Non-Destructive Overrides}: Stronger layers override without modifying original data
\end{itemize}

\section{Formal Complexity Analysis}

\subsection{Current Architecture Costs}

Transformer attention operates over both feature and sequence dimensions:

\begin{equation}
\text{Attention cost} = O(n^2 d) \text{ per layer}
\end{equation}

\begin{equation}
\text{Total inference} = O(L \cdot n^2 d) \text{ for } L \text{ layers}
\end{equation}

For typical values ($n=8192$, $d=4096$, $L=32$), a single forward pass involves approximately $10^{13}$ operations. This occurs on \textit{every inference}, regardless of whether the query involves known or novel information.

\subsection{Hypothesized Persistent-State Costs}

\begin{table}[H]
\centering
\caption{Complexity Comparison}
\begin{tabular}{lll}
\toprule
\textbf{Operation} & \textbf{Transformer} & \textbf{USD Substrate} \\
\midrule
Direct fact lookup & $O(L \cdot n^2 d)$ & $O(1)$ path traversal \\
Relationship query & $O(L \cdot n^2 d)$ & $O(e)$, $e$ = edge count \\
Context composition & $O(L \cdot n^2 d)$ & $O(k)$, $k$ = prims loaded \\
Knowledge update & Full retraining & $O(1)$ opinion insertion \\
\bottomrule
\end{tabular}
\end{table}

\subsection{Theoretical Energy Ratio}

For cached knowledge retrieval:

\begin{equation}
\text{Energy Ratio} = \frac{O(L \cdot n^2 d)}{O(1)} = O(L \cdot n^2 d)
\end{equation}

With typical parameters, this suggests a theoretical maximum speedup of $10^{13}$ for direct fact retrieval. Our hypothesis of $>$10$\times$ is extremely conservative.

\begin{figure}[htbp]
    \centering
    \includegraphics[width=\textwidth]{figures/figure5-energy.pdf}
    \caption{Energy Cost Comparison: Transformer vs USD Substrate. For direct
             fact lookup queries, transformers require $O(L \cdot n^2 d)$ operations
             ($\sim 10^{13}$ ops), while USD achieves $O(1)$ via path traversal.
             Relationship queries reduce from $O(L \cdot n^2 d)$ to $O(e)$ (edge count).
             Knowledge updates require full retraining (hours to days) in transformers
             but only $O(1)$ opinion insertion in USD. This demonstrates exponential
             energy savings for persistent state operations.}
    \label{fig:energy}
\end{figure}

\section{The Persistent State Hypothesis}

\subsection{Formal Statement}

\begin{quote}
\textbf{Hypothesis}: The emergent capabilities of large-scale neural networks (reasoning, analogy, generalization) can be preserved in a persistent, composable substrate that does not require constant recomputation. A well-designed persistent-state architecture could achieve \textbf{$>$10$\times$ energy reduction} for retrieval of known knowledge while maintaining \textbf{$>$80\% capability preservation} for reasoning tasks.
\end{quote}

\subsection{Energy Sinks in Current Architectures}

\begin{table}[H]
\centering
\caption{Energy Sinks and Architectural Causes}
\begin{tabular}{ll}
\toprule
\textbf{Energy Sink} & \textbf{Architectural Cause} \\
\midrule
No persistent state & Every inference recomputes from scratch \\
$O(n^2)$ attention & Context length explodes compute quadratically \\
No incremental learning & Cannot add knowledge---must retrain \\
Redundant pattern matching & Re-derives identical relationships per query \\
Monolithic weights & Cannot selectively load relevant knowledge \\
\bottomrule
\end{tabular}
\end{table}

\subsection{The Compilation Metaphor}

\begin{itemize}
    \item \textbf{Interpretation}: Execute source code directly. High flexibility, high runtime cost.
    \item \textbf{Compilation}: Transform to optimized representation once, execute cheaply many times.
\end{itemize}

Current LLMs are pure interpreters. A persistent-state architecture enables \textit{knowledge compilation}: expensive inference happens once, results persist, retrieval is cheap.

\section{USD Semantic Mapping}

\begin{table}[H]
\centering
\caption{USD to Cognitive Mapping}
\begin{tabular}{lll}
\toprule
\textbf{USD Concept} & \textbf{Scene Graph Function} & \textbf{Cognitive Analog} \\
\midrule
Prims & Addressable units & Knowledge fragments \\
Composition Arcs & Layer and combine & Selective knowledge loading \\
Payloads & Deferred loading & Lazy context evaluation \\
Opinions & Non-destructive overrides & Incremental learning \\
Layer Stacking & Additive modifications & Build on prior reasoning \\
Time Samples & Temporal state access & Memory without re-inference \\
Variant Sets & Switchable alternatives & Hypothesis navigation \\
LIVRPS Resolution & Deterministic conflict handling & Knowledge arbitration \\
\bottomrule
\end{tabular}
\end{table}

\textbf{Key Insight: Navigation vs. Loading}

USD treats scene data as an \textit{external environment to navigate} rather than content to load entirely into memory. We hypothesize cognitive architectures could similarly treat knowledge as an external environment, loading only task-relevant fragments via graph traversal.

\section{Preliminary Results: USD Cognitive Substrate}

\subsection{Demonstrated Mechanisms}

\begin{table}[H]
\centering
\caption{Implementation Status}
\begin{tabular}{lll}
\toprule
\textbf{Mechanism} & \textbf{Implementation} & \textbf{Status} \\
\midrule
LIVRPS Composition & Session $>$ calibration $>$ profile & \textbf{Demonstrated} \\
Selective Loading & Domain payloads load on demand & \textbf{Demonstrated} \\
Layered Opinions & Hebbian weight updates preserve baseline & \textbf{Demonstrated} \\
Deterministic Routing & Batch-invariant inference integration & \textbf{Demonstrated} \\
Temporal Compilation & Session $\rightarrow$ daily $\rightarrow$ weekly $\rightarrow$ calibration & \textbf{Demonstrated} \\
Context Restoration & Staleness-aware snapshot retrieval & \textbf{Demonstrated} \\
\bottomrule
\end{tabular}
\end{table}

\subsection{Limitations}

The implementation manages \textit{behavioral state} (cognitive mode, energy level, momentum phase), not \textit{factual knowledge}. The demonstrated mechanisms prove the pattern works, but do not validate the full hypothesis.

\section{Uncertainty Calibration}

\subsection{Probability Estimates}

\begin{table}[H]
\centering
\caption{Outcome Probability Estimates}
\begin{tabular}{lr}
\toprule
\textbf{Outcome} & \textbf{Estimated Probability} \\
\midrule
Pattern extends to factual knowledge cleanly & 60--70\% \\
Capability preservation (partial) & 40--60\% \\
Capability preservation (full) & 30--50\% \\
Energy savings $>$10$\times$ & 30--50\% \\
\textbf{Full hypothesis validation} & \textbf{30--50\%} \\
Valuable learnings even if refuted & $>$90\% \\
\bottomrule
\end{tabular}
\end{table}

\section{Falsification Criteria}

\subsection{The hypothesis should be considered REFUTED if:}

\begin{enumerate}
    \item Energy savings $<$2$\times$ (architectural benefit is marginal)
    \item Capability degradation $>$50\% (distillation loses too much)
    \item Composition incoherence (combined fragments produce nonsense)
    \item Scale failure (architecture breaks at realistic knowledge sizes)
    \item Query parsing failure (cannot achieve reasonable semantic mapping)
\end{enumerate}

\subsection{The hypothesis should be considered VALIDATED if:}

\begin{enumerate}
    \item Energy savings $>$10$\times$ for retrieval of known knowledge
    \item Capability preservation $>$80\% for reasoning tasks
    \item Composition coherence produces useful answers
    \item Scale works for realistic knowledge graph sizes ($>$100K prims)
    \item Graceful degradation to neural inference for novel queries
\end{enumerate}

\section{Conclusion}

We have presented the Persistent State Hypothesis, a challenge to the industry assumption that intelligence necessarily requires proportional energy consumption.

Preliminary results from the USD Cognitive Substrate demonstrate that USD's composition mechanisms successfully manage cognitive behavioral state. Whether these mechanisms extend to factual knowledge retrieval---and whether emergent capabilities survive the transition---remains the core research question.

\textbf{The AI industry is betting trillions on the assumption that intelligence requires energy. We propose a different bet: that the energy problem is architectural, and that USD semantics might inform a more efficient path.}

\begin{quote}
\textit{``To invent something is about 100 times harder than it is to copy it.''} ---Demis Hassabis, January 2026
\end{quote}

We are attempting invention.

\section*{Acknowledgments}

The author thanks Pixar Animation Studios for the USD framework that inspired this architecture, and the ThinkingMachines Lab for demonstrating that LLM determinism is achievable through batch-invariant design \cite{thinkingmachines2025}.

\section*{Code and Data Availability}

The USD Cognitive Substrate specification and reference implementation are available under the MIT License:

\begin{itemize}
    \item \textbf{Specification:} \url{https://github.com/JosephOIbrahim/usd-cognitive-substrate}
    \item \textbf{Implementation:} \url{https://github.com/JosephOIbrahim/Orchestra}
\end{itemize}

\bibliographystyle{plainnat}
\bibliography{../references}

\end{document}
